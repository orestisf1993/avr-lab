Breakpoints εισήχθησαν στα εξής σημεία:
\begin{enumerate}
\item Μετά το τέλος της ρουτίνας αρχικών συνθηκών
(κάτω από το σχόλιο \lstinline[breaklines=true]!;load aem1 in memory!) για να ελέγξουμε ότι οι καταχωρητες
\lstinline!aem1L!,\lstinline!aem1H!,\lstinline!aem2L! και \lstinline!aem2H! πήραν την σωστή τιμή.
\item Μετά το σχόλιο \lstinline[breaklines=true]!;define PORTB as exit! για να ελέγξουμε την σωστή
αρχικοποίηση της θύρας Β και στην συνέχεια εκτελώντας μια-μια τις
εντολές στις συναρτήσεις \lstinline!show_aem1_digits! και \lstinline!show_aem2_digits!
βλέπουμε την σωστή ένδειξη των ΑΕΜ στα αντίστοιχα LEDs.
(Χρησιμοποιήσαμε μια προσωρινή ρουτίνα για να περάσει την εντολη delay σε ένα κύκλο)
\item Στην αρχή του δεύτερου μέρους της άσκησης για να εκτελέσουμε μια-μια τις εντολές. 
Αρχικά, χωρίς να πατάμε κανένα switch για να ελέγξουμε ότι γίνεται σωστά η επαναληπτική διαδικασία.
Στην συνέχεια, με την επιλογή ενός εκ των τριών switch ελέγχουμε αν το πρόγραμμα πάει στην σωστή ρουτίνα (ανάλογα το switch).
Τέλος, δεν ορίσαμε κάποιο breakpoint αλλά συνεχίσαμε να εκτελούμε εντολή-εντολή πατώντας και απελευθερώνοντας κάποιο switch
για έλεγχο στην συνέχεια της ένδειξης των led.
\end{enumerate}