\chapter{Τμήμα 1}

Σκοπός του πρώτου τμήματος της εργαστηριακής άσκησης είναι η εισαγωγή και εμφάνιση των ΑΕΜ στα led της κάρτας STK500.
Αρχικά, αποθηκεύονται τα ΑΕΜ σε μορφή BCD σαν constant words με την χρήση της εντολής \lstinline!.dw! και φορτώνονται στην μνήμη μέσω των εντολών \lstinline!lpm! και \lstinline!ldi! χρησιμοποιώντας τους Z καταχωρητές.
\begin{lstlisting}
;define student's info
aem1dw: .dw $7796
aem2dw: .dw $7853

main:

.def aem1H=R6
.def aem1L=R7
.def aem2H=R8
.def aem2L=R9
.def temp_carry=r24

...

;load aem1 in memory
ldi ZH,high(2 * aem1dw)
ldi ZL,low(2 * aem1dw)
lpm aem1L, Z+
lpm aem1H, Z

;load aem2 in memory
ldi ZH,high(2 * aem2dw)
ldi ZL,low(2 * aem2dw)
lpm aem2L, Z+
lpm aem2H, Z
\end{lstlisting}

Στη συνέχεια, φορτώνουμε ένα-ένα τα ψηφία των ΑΕΜ στους καταχωρητές \lstinline!r16! με \lstinline!r23! και αποθηκεύουμε το άθροισμα τους στους καταχωρητές \lstinline!r16! με \lstinline!r19!
\begin{lstlisting}
;aem2 in digits
ldi r16,0x03
ldi r17,0x05
ldi r18,0x08
ldi r19,0x07

;aem1 in digits
ldi r20,0x06
ldi r21,0x09
ldi r22,0x07
ldi r23,0x07

; add digits one by one
add r16,r20
add r17,r21
add r18,r22
add r19,r23
\end{lstlisting}

Ελέγχουμε από το τέλος για κάθε ψηφίο αν υπάρχει carry (δηλαδή να είναι μεγαλύτερο από το $10$)
και αν ναι, αφαιρούμε $10$ και προσθέτουμε ένα στον επόμενο ψηφίο.

\begin{lstlisting}
cpi r16,0x0A
; if r16 is smaller than 10 jump to next comparison
brlo cmp17
subi r16,0x0A
subi r17,-1

cmp17:
cpi r17,0x0A
brlo cmp18
subi r17,0x0A
subi r18,-1

cmp18:
cpi r18,0x0A
brlo cmp19
subi r18,0x0A
subi r19,-1

cmp19:
cpi r19, 0x0A
brlo cmpend
subi r19,0x0A

cmpend:
\end{lstlisting}

Αν το τελευταίο (πιο σημαντικό) ψηφίο είναι μεγαλύτερο από το $10$ τότε αφαιρούμε $10$ και το αφήνουμε ως έχει. Δηλαδή, για τα ΑΕΜ $7796$ και $7853$ θεωρούμε ότι το αποτέλεσμα είναι $5649$ και όχι $F649$.

Το τελικό αποτέλεσμα φορτώνεται στις θέσεις μνήμης \lstinline!0x0060! με \lstinline!0x0064! της SRAM.
\begin{lstlisting}
; move digits to SRAM
sts 0x0060, r19
sts 0x0061, r18
sts 0x0062, r17
sts 0x0063, r16
\end{lstlisting}

Για την έξοδο στα led χρησιμοποιούμε το PORTB και του θέτουμε την τιμή \lstinline!0xFF! για να το ορίσουμε ως έξοδο.
\begin{lstlisting}
;define PORTB as exit
.def temp=r25
ser temp
out DDRB,temp
\end{lstlisting}

Για την εμφάνιση των ψηφίων χρησιμοποιούνται οι ρουτίνες
\lstinline!show_aem1_digits! και \lstinline!show_aem2_digits!
Σε κάθε στιγμή απεικονίζονται 2 ψηφία στα leds, 4 leds χρησιμοποιούνται για κάθε ψηφίο.
Στη συνέχεια, η ρουτίνα
\lstinline!delay10! χρησιμοποιείται για την δημιουργία χρονικής καθυστέρησης 10 δευτερολέπτων για συχνότητα ρολογιού 4MHz.
Η εντολή \lstinline!com! χρησιμοποιείται για την αντιστροφή των bit των καταχωρητών (συμπλήρωμα του $1$) καθώς η κάρτα ST500 χρησιμοποιεί αντίστροφη λογική για την χρήση των led.
\begin{lstlisting}
show_aem1_digits:
    ; load on portb a 8-bit register with 2 last digits
    ldi r18, 0x06
    ldi r19, 0x09
    swap r19
    or r19,r18
    ; r19 was 1 for active led but st500 has inverted logic
    ; because of that we use 1-complement
    com r19
    out PORTB,r19
    rcall delay10
    ldi r18, 0x07
    ldi r19, 0x07
    swap r19
    or r19,r18
    com r19
    out PORTB,r19
    rcall delay10
    ret

show_aem2_digits:
    ldi r18, 0x03
    ldi r19, 0x05
    swap r19
    or r19,r18
    com r19
    out PORTB,r19
    rcall delay10
    ldi r18, 0x08
    ldi r19, 0x07
    swap r19
    or r19,r18
    com r19
    out PORTB,r19
    rcall delay10
    ret

; Delay 40 000 000 cycles
; 10s at 4 MHz
delay10:
    ldi  r18, 203
    ldi  r19, 236
    ldi  r20, 133
delay10L1: dec  r20
    brne delay10L1
    dec  r19
    brne delay10L1
    dec  r18
    brne delay10L1
    ret
\end{lstlisting}

Τελικά, εμφανίζονται και τα ψηφία του αθροίσματος με τον ίδιο τροπο.
\begin{lstlisting}
; show digits from addition
lds r19,0x0062
lds r18,0x0063
swap r19
or r19,r18
com r19
out PORTB,r19
rcall delay10

lds r19,0x0060
lds r18,0x0061
swap r19
or r19,r18
com r19
out PORTB,r19
rcall delay10
\end{lstlisting}