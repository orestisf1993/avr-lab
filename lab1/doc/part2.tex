\chapter{Τμήμα 2}

Στο δεύτερο τμήμα της εργαστηριακής άσκησης το πρόγραμμα παραμένει σε ένα βρόχο αναμονής έως ότου πατηθούν
και απελευθερωθούν οι διακόπτες SW1,SW2,SW3.
Με τον SW1 εμφανίζονται στα LED's τα ψηφία του πρώτου ΑΕΜ με τον ίδιο τρόπο όπως στο τμήμα 1.
Με τον SW2 εμφανίζονται στα LED's τα ψηφία του δεύτερου ΑΕΜ.
Με διαδοχικά πατήματα του SW3 εμφανίζονται στα LED0-LED3 όλα τα ψηφία του αθροίσματος σε μορφή BCD των 2 ΑΕΜ.
Αρχικά, ορίζουμε τον PORTB ως έξοδο όπως στο τμήμα 1.
Σαν είσοδο των σημάτων ελέγχου στους διακόπτες χρησιμοποιούμε το PORTD
και του θέτουμε την τιμή \lstinline!0x00! για να το ορίσουμε ως είσοδο.
\begin{lstlisting}
;PART2
;define PORTD as input
ser temp
out PORTB,temp
clr temp
out DDRD,temp
\end{lstlisting}

Στη συνέχεια φορτώνουμε στον καταχωρητή r22,τα σήματα ελέγχου που δόθηκαν στο PORTD και ελέγχουμε για κάθε SW με την εντολή \lstinline!sbrs!
αν το σήμα που αναφέρεται στον καθένα είναι $1$.
Aν ναι τότε παρακάμπτει την κλήση ρουτίνας για την εμφάνιση των
ψηφίων (καθώς η κάρτα ST500 χρησιμοποιεί αντίστροφη λογική για την
εμφάνιση των LEDs) και ελέγχει με τον ίδιο τρόπο το σήμα για τα επόμενα SW.
\begin{lstlisting}
;wait in this loop until the user press a switch
wait_loop:
in r22,PIND
sbrs r22,1
rcall sw1_pressed
sbrs r22,2
rcall sw2_pressed
sbrs r22,3
rcall sw3_pressed
jmp wait_loop
ret
\end{lstlisting}

Σε περίπτωση που το SW1 έχει τιμή $0$ (άρα είναι πατημένο) καλείται η ρουτίνα \lstinline!sw1_pressed!.
Mέσα σε αυτήν την ρουτίνα υπάρχει εμφωλευμένη η \lstinline!sw1_loop!
η οποία καλείται για όσο το SW1 έχει τιμή 0 (άρα εξακολουθεί να είναι πατημένο).
Όταν το αντίστοιχο πλήκτρο απελευθερωθεί και άρα το sw1 πάρει τιμή $1$
καλείται η συνάρτηση \lstinline!show_aem1_digits! που χρησιμοποιήσαμε στο part1.
\begin{lstlisting}
;wait until user releases switch 1
sw1_pressed:
sw1_loop:
 in r22,PIND
 sbrs r22,1
 jmp sw1_loop
 rcall show_aem1_digits
 ret
Σε περίπτωση που το Sw2 έχει τιμή 0 (αρα 1 στο led) καλείται η ρουτίνα
sw2_pressed και ακολουθείται η ίδια λογική.
;wait until user releases switch 2
sw2_pressed:
sw2_loop:
 in r22,PIND
 sbrs r22,2
 jmp sw2_loop
 rcall show_aem2_digits
 ret
\end{lstlisting}

Σε περίπτωση που το SW3 έχει τιμή $0$ καλείται η ρουτίνα
\lstinline!sw3_pressed!.
Μέσα σε αυτήν υπάρχει εμφωλευμένη η \lstinline!digits0! στην οποία το
πρόγραμμα επανέρχεται για όσο το SW3 έχει τιμή $0$).
Όταν το πλήκτρο απελευθερωθεί καλείται η συνάρτηση \lstinline!show_addition_digit1!.
\begin{lstlisting}
;wait until user releases switch 3
sw3_pressed:
digits0:
	in r22,PIND
	sbrs r22,3
	jmp digits0
	rcall show_addition_digit1
\end{lstlisting}
Στην \lstinline!show_addition_digit1! φορτώνουμε από την SRAM στον καταχωρητή \lstinline!r19!
το 1ο ψηφίο του αθροίσματος των 2 ΑΕΜ που βρίσκεται στην \lstinline!0x0060!.
Χρησιμοποιούμε την εντολή com για την αντιστροφή των leds και το εμφανίζουμε στο PORTB με την εντολή
\lstinline!out!.
\begin{lstlisting}
show_addition_digit1:
	lds r19,0x0060
	com r19
	out PORTB,r19
	ret
\end{lstlisting}

Μετά την εκτέλεση της συνάρτησης \lstinline!show_addition_digit1! το πρόγραμμα
μπαίνει στην ρουτίνα \lstinline!pressed1!.
Αυτή καλείται για όσο η τιμή στον SW3 είναι $1$.
Όταν η τιμή γίνει $0$ (άρα πατηθεί το πλήκτρο SW3 ξανά)
το πρόγραμμα πάει στην ρουτίνα \lstinline!digits1! όπου εκεί με τον ίδιο κωδικά όπως στο \lstinline!digits0!
περιμένει μέχρι να απελευθερωθεί το SW3 για να κληθεί η \lstinline!show_addition_digit2! για την εμφάνιση του 2ου ψηφίου.
Αυτό επαναλαμβάνεται για την εμφάνιση όλων των ψηφίων.
\begin{lstlisting}
pressed1:
 in r22,PIND
 sbrc r22,3
 jmp pressed1
digits1:
 in r22,PIND
 sbrs r22,3
 jmp digits1
 rcall show_addition_digit2
pressed2:
 in r22,PIND
 sbrc r22,3
 jmp pressed2
digits2:
 in r22,PIND
 sbrs r22,3
 jmp digits2
 rcall show_addition_digit3
pressed3:
 in r22,PIND
 sbrc r22,3
 jmp pressed3
digits3:
 in r22,PIND
 sbrs r22,3
 jmp digits3
 rcall show_addition_digit4
 ret
\end{lstlisting}

Οι συναρτήσεις \lstinline!show_addition_digit2!,\lstinline!3!,\lstinline!4!:
\begin{lstlisting}
show_addition_digit2:
 lds r19,0x0061
 com r19
 out PORTB,r19
 ret
show_addition_digit3:
 lds r19,0x0062
 com r19
 out PORTB,r19
 ret
show_addition_digit4:
 lds r19,0x0063
 com r19
 out PORTB,r19
 ret
\end{lstlisting}